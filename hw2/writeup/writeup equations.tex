% docx2tex 1.8 --- ``Who let the docx out?'' 
% 
% docx2tex is Open Source and  
% you can download it on GitHub: 
% https://github.com/transpect/docx2tex 
%  
\documentclass{scrbook} 
\usepackage[T1]{fontenc} 
\usepackage[utf8]{inputenc} 
\usepackage{hyperref} 
\usepackage{multirow} 
\usepackage{tabularx} 
\usepackage{color} 
\usepackage{textcomp} 
\usepackage{tipa}
\usepackage{amsmath} 
\usepackage{amssymb} 
\usepackage{amsfonts} 
\usepackage{amsxtra} 
\usepackage{wasysym} 
\usepackage{isomath} 
\usepackage{mathtools} 
\usepackage{txfonts} 
\usepackage{upgreek} 
\usepackage{enumerate} 
\usepackage{tensor} 
\usepackage{pifont} 
\usepackage{ulem} 
\usepackage{xfrac} 
\usepackage{arydshln} 
\usepackage[english]{babel}

\begin{document}
Planar Homography

\textbf{Part (a)}

For a point $w$ on the common plane, $\tilde{u}$ and $\tilde{x}$ are the 2D projected point on the first and second camera respectively. The two are related by a relationship
\begin{equation*}
\lambda \tilde{x}=H\tilde{u}
\end{equation*}
For $N$ pairs of such points, we have

\begin{align*}
\lambda _{n}\widetilde{x_{n}}&=H\widetilde{u_{n}} \\
\lambda _{n}\left[\begin{array}{c}
\widetilde{x_{xn}}\\
x_{yn}\\
1
\end{array}\right]&=\begin{bmatrix}
H_{11} & H_{12} & H_{13}\\
H_{21} & H_{22} & H_{23}\\
H_{31} & H_{32} & H_{33}
\end{bmatrix}\left[\begin{array}{c}
\widetilde{u_{xn}}\\
\widetilde{u_{yn}}\\
1
\end{array}\right]=\left[\begin{array}{c}
-{h}_{1}^{T}-\\
-{h}_{2}^{T}-\\
-{h}_{3}^{T}-
\end{array}\right]\widetilde{u_{n}} 
\end{align*}

We let $h=\left[\begin{array}{cccc}
H_{11} & H_{12} & \ldots & H_{33}
\end{array}\right]^{T}$. Then for one set of points,

\begin{align*}
\lambda _{n}\widetilde{x_{xn}}&={h}_{1}^{T}\widetilde{u_{n}} \\
\lambda _{n}\widetilde{x_{yn}}&={h}_{2}^{T}\widetilde{u_{n}} \\
\lambda _{n}&={h}_{3}^{T}\widetilde{u_{n}} 
\end{align*}

which can be reduced to

\begin{align*}
{h}_{1}^{T}\widetilde{u_{n}}-\left({h}_{3}^{T}\widetilde{u_{n}}\right)\widetilde{x_{xn}}&=0 \\
{h}_{2}^{T}\widetilde{u_{n}}-\left({h}_{3}^{T}\widetilde{u_{n}}\right)\widetilde{x_{yn}}&=0 
\end{align*}

In matrix form,
\begin{equation*}
A_{n}=\left[\begin{array}{ccc}
\widetilde{u_{n}}^{T} & 0 & -\widetilde{x_{xn}}\widetilde{u_{n}}^{T}\\
0 & \widetilde{u_{n}}^{T} & -\widetilde{x_{yn}}\widetilde{u_{n}}^{T}
\end{array}\right]\left[\begin{array}{c}
h_{1}\\
h_{2}\\
h_{3}
\end{array}\right]
\end{equation*}
Repeat this $N$ times to get
\begin{equation*}
A=\left[\begin{array}{c}
A_{1}\\
A_{2}\\
\vdots \\
A_{N}
\end{array}\right]
\end{equation*}
Part (b)

There are 9 elements

\textbf{Part (c) NOT YET DONE}

$H$ is a matrix that encodes a projective transformation. Hence, it has 7 degrees of freedom.

Each point correspondence gives two linear equations (derived above).

Part (d)

1. Compute the SVD of $A$ to get $U\Upsigma V^{T}$.

2. The solutions for $h$ is given by the rightmost column of the matrix $V$.
\end{document}
